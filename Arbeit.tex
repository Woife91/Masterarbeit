% Arbeit.tex
% LaTeX-Hauptdatei fuer Studien/Diplomarbeiten am IMMD 9
% geschrieben von Wolfgang Heidrich <wgheidri@immd9.informatik.uni-erlangen.de>
% erweitert von Christian Vogelgsang <cnvogelg@immd9.informatik.uni-erlangen.de>
% benoetigt LaTeX 2e (z.B. in teTeX)

% --- Style + Optionen ---
% Font: 11pt bevorzugt, 10pt fuer besonders lange Arbeiten.
%       12pt nur in Ausnahmefaellen.
\documentclass[11pt, twoside, openright]{studdipl} 

% --- Paketauswahlt ---
% a4wide: breites Papierformat
% german: Deutsche Ueberschriften
% epsfig: figures mit EPS Bilder
\usepackage{a4wide,german}
\usepackage{graphicx}
\usepackage{color}

%%%%%%%%%%%%%%%%%%%%%%%%%%%%%%%%%%%%%%%%%%%%%%%%%%%%%%%%%%%%%%
%\usepackage[pdftex]{graphicx}
\DeclareGraphicsExtensions{.pdf,.png,.jpg}
\usepackage[german]{babel} %english language support
\usepackage[T1]{fontenc}
\usepackage{a4wide} % enlarges the A4 printing area
\usepackage{abbrevs} % defines abbreviation macros
\usepackage{amsmath} % mathematic formulas
\usepackage{amssymb} % mathematic special characters
\usepackage{dsfont}  % double stroke font (e.g. for mathematical sets)
\usepackage{ifthen}
\usepackage{float} % 3 different styles of float environments (tables or figures): plain, boxed and ruled
\usepackage{color} % displaying colors
\definecolor{mygray}{rgb}{0.5,.5,.5} % color define
\usepackage[absolute]{textpos} % absolute positioning of text
\usepackage[amssymb,thinqspace]{SIunits} % formatting support for the Systeme International d?Unites (SI)
\usepackage[printonlyused]{acronym} % management of acronyms and acronym lists (only used acronyms are printed in the list)
\usepackage{listings}  % embed listings (must be included before the package algorithm!
\usepackage{fancyhdr}% for funny headlines and footers
\usepackage{multirow} % tune tables
\usepackage[hang,small,bf]{caption} % smaller captions
\usepackage{vmargin} % pageaylout - borders
\usepackage{setspace} % Zeilenabst�nde..
\usepackage{titlesec} % formatting of sections etc.
\usepackage[colorlinks=true,
            citecolor=black,
            linkcolor=black,   % color of hyperref links
            urlcolor=black,       % color of page of \url{...}
            breaklinks=true    % break links if exceeding a single line
]{hyperref}
\usepackage[numbers]{natbib}
\usepackage{natbib}
\usepackage{engord} %english ordered numbers
\usepackage{soul}
\usepackage{captcont}

\pdfcompresslevel=0
\pdfoutput=1
\pdfcatalog{ /PageMode   /UseOutlines}

\usepackage[absolute]{textpos}
\usepackage{pdfpages}
\usepackage{csvsimple}


\newcommand{\citeeig}[1]{\citeauthor{#1} \cite{#1}}
\newcommand{\trade}{\textsuperscript{\textregistered}}
%%%%%%%%%%%%%%%%%%%%%%%%%%%%%%%%%%%%%%%%%%%%%%%%%%%%%%%%%%%%%%%%%%

% --- CV Config: ---

% --- weitere Pakete ---

% inputenc: direkte Eingabe von Umlauten erlaubt!
\usepackage[isolatin]{inputenc}
% huebsche Rahmen fuer Sourcecodebloecke
\usepackage{fancybox}
% erlaubt innerhalb eines Figure blocks mehrere Subfigures
\usepackage{subfig}  % urspruenglich subfigure
% verwende PostScript-Fonts
%\usepackage{pslatex}
\usepackage{times}
%\usepackage{dfadobe}           % ditto

% --- neue Environments ---

% \begin{cvSrcBox}.....\end{cvSrcBox} 
% Dies erzeugt eine Verbatim-Box mit einem Rahmen drumherum.
% Sie ist gut fuer Quelltextausschnitte geeignet!
\newenvironment{cvSrcBox}%
{\VerbatimEnvironment \begin{Sbox}\begin{minipage}{15cm}\begin{Verbatim}}%
{\end{Verbatim}\end{minipage}\end{Sbox}\setlength{\fboxsep}{3mm}\fbox{\TheSbox}}

% --- neue Kommandos ---

% \cvFig <file> <caption>
% Erzeugt eine Figure Umgebung mit einer xfig-Darstellung
% Die .fig Datei muss im Verzeichnis "figures" liegen
% Es wird automatisch ein Label mit dem Namen der Date erzeugt!
% Den Dateinamen ohne Endung .fig angeben!
\newcommand{\cvFig}[2]{\begin{figure}\begin{center}\input{figures/#1.tex}\end{center}\caption{\label{#1}#2}\end{figure}}

% \cvPic [width] <file> <caption>
% Erzeugt analog wie cvFig eine figure, doch diesmal mit einem
% Bild. Das Bild wird als PNG Datei in "pictures" abgelegt und
% dort automatisch nach EPS konvertiert.
\newcommand{\cvPic}[3][7cm]{\begin{figure}\begin{center}\includegraphics[width=#1]{pictures/#2.ps}\end{center}\caption{\label{#2}#3}\end{figure}}

% \cvPlot [width] <file> <caption>
% Erzeugt analog wie cvFig eine figure, doch diesmal mit einem
% Bild. Das Bild wird von GNUplot aus einer .plt-Datei erzeugt
\newcommand{\cvPlot}[3][10cm]{\begin{figure}\begin{center}\includegraphics[width=#1]{plots/#2.ps}\end{center}\caption{\label{#2}#3}\end{figure}}

% --- Optionen ---
% steuert das Figure Placement auf den Seiten
\renewcommand{\floatpagefraction}{0.8}

% definiert die Kopfzeile
\lhead[]{\fancyplain{}{\rightmark}}
\rhead[{\fancyplain{}{\leftmark}}]{}

% --- CV Config Ende ---

% Ein wenig liberalere spacing rules
\frenchspacing

% Erlaube groessere Freiraeume zwischen Woertern.
% (wichtig fuer Deutsche Texte wegen der grossen durchschnittlichen
% Wortlaenge). Fuer Englische Arbeiten moeglicherweise weglassen.
\sloppy

% -------------- Konfiguration ------------------------------------------------
% hier werden die individuellen Parameter der Arbeit festgelegt
% -> ALSO AENDERN!

% Typ der Arbeit auswaehlen:
%\thesistype{Studienarbeit}
\thesistype{Masterarbeit}

% Titel der Arbeit
\title{Farbnormalisierung f�r die automatische Klassifikation von Knochenmarkzellen}

% AutorIn <- Dein Name :-)
\author{Wolfgang Aichinger}

% Dein Geburtsdatum
\birthday{16. April 1991}

% Dein Geburtsort:
\birthplace{Deggendorf}

% DeinE BetreuerIn:
\supervisor{Dipl.-Inf. Sebastian Krappe}

% Beginn der Arbeit
\bdate{02. November 2015}

% Abgabetermin
\edate{02. Mai 2016}

% -------------- Ende der Konfiguration ---------------------------------------

\setcounter{secnumdepth}{3}
\setcounter{tocdepth}{5}

\begin{document}

% DRAFT MODE
% Erzeugt eine Ueberschrift mit dem Datum des Drafts. Muss fuer die
% endgueltige Version natuerlich auskommentiert werden!!!
%\draft

% Der "Vorspann" hat roemische Seitennummern 
\prepages

% Damit wird die zweite Titelseite erstellt (die erste ist ja in einem
% separaten File)
\maketitle

% eine Leerseite
\vspace*{10cm} \newpage

% Inhaltsverzeichnis
\tableofcontents\newpage
% Verzeichnis aller Zeichnungen - optional
\listoffigures\newpage
% Verzeichnis aller Tabellen - optional
\listoftables\newpage

% eine Leerseite
\vspace*{10cm} \newpage

% der eigentliche Text hat arabische Nummern
\mainbody

% sorgt dafuer, dass alle Eintraege der Literature.bib im 
% Literaturverzeichnis aufgefuehrt werden
\nocite{*}

% ---------- Kapitel ---------
% Die Datei Chapter.tex wird automatisch erzeugt!

%\setlength{\parskip}{-0.5ex plus1ex}

\newpage
\pagenumbering{arabic}
\pagestyle{fancy}
\chapter[Einleitung]{Einleitung}\label{sec:einleitung}
%\begin{tikzpicture} 
%	\begin{axis}[ 
%		xlabel=Bild, 
%		ylabel=Metrik,
%	 	axis x line*=bottom,
%	 	axis y line*=left,
%	 	ymajorgrids,
%	 	ymin=0,
%	 	ymax=1
%	] 
%	 \addplot [
%	 	color=blue,
%       ultra thick,
%      point meta=x, % Define the value that's used to determine the color
%        mark=x
%    ] coordinates { 
%	(1, 1) 
%	(2, 0.689128518) 
%	(3, 0.313492749)
%	(4, 0.197890621)
%	(5, 0.155663954)
%	(6, 0.135560304)
%	(7, 0.125045233)
%	(8, 0.119600855)
%	(9, 0.114321953)
%	(10, 0.112258035)
%	}; 
%	\end{axis} 
%\end{tikzpicture}



\section[Motivation]{Motivation}\label{sec:motivation}
Knochenmarkanalysen werden im klinischen Alltag in den verschiedensten F�llen angefordert, z.B. bei Verdacht auf Leuk�mie, da es sich hierbei um eine Erkrankung des Knochenmarks handelt. Die Krankheit kann in allen Altersgruppen auftreten, es wird zwischen den folgenden, h�ufigsten, Formen je nach Verlauf und urs�chlichem Zelltyp unterschieden:
\begin{itemize}
\item{Akute myeloische Leuk�mie (AML)}
\item{Chronische myeloische Leuk�mie (CML)}
\item{Akute lymphatische Leuk�mie (ALL)}
\item{Chronische lymphatische Leuk�mie (CLL)}
\end{itemize}
Im Kindesalter tritt fast immer ein akuter Verlauf auf, am h�ufigsten wird eine ALL diagnostiziert. Unabh�ngig vom Alter bildet jedoch die CLL mit �ber einem Drittel die zahlreichste Form. Im Jahr 2012 lag die Zahl der Neuerkrankungen an Leuk�mie bei ca. 12640 F�llen, 5\% davon unter 15 Jahren. Eine �bersicht �ber das Alter der neuerkrankten Patienten zeigt Abbildung \ref{fig:leukemia_age}. Aus dieser Statistik geht ebenfalls hervor, dass vor allem im Erwachsenenalter M�nner h�ufiger betroffen sind als Frauen.
\begin{figure}
\center
\includegraphics[width = 0.9\textwidth]{pics/Einleitung/leukaemie_statistic_age}
\caption[Neuerkrankungen Leuk�mie nach Alter und Geschlecht]{Die Graphik zeigt die Statistik der Neuerkrankungen mit Leuk�mie in Deutschland f�r die Jahre 2011 und 2012 nach Alter (horizontale Achse) und Geschlecht (Balkenfarbe). Die Zahlen sind dabei relativ je 100.000 normiert \cite{url_leukemia_statistic}.\label{fig:leukemia_age}}
\end{figure}
Bei Kindern ist die Prognose deutlich g�nstiger als bei Erwachsenen. Die 5-Jahres �berlebenschance wird f�r sie mit bis zu 90\% angegeben, der Wert f�r die Erwachsenen liegt nur bei 35-50\%. Eine dauerhafte Heilung ist m�glich, es gibt sie aber nur selten z.B. durch Stammzellentransplantation. �ber die Ursachen von Leuk�mie ist indes wenig bekannt. Als Risikofaktoren gelten ionisierende Strahlung, z.B. bei Strahlen- oder Chemotherapie, und der Umgang mit Chemikalien wie Benzol. Diese Faktoren treffen aber nur auf eine kleine Gruppe der Erkrankten zu, weshalb weitere Faktoren diskutiert werden, ohne dass diese bis jetzt nachgewiesen werden konnten. Hierzu geh�ren Ern�hrungsgewohnheiten und Lebensstil, vor allem beim chronischen Verlauf, ein Einfluss von Viren und ein ungen�gendes Training des Immunsystem bei Heranwachsenden \cite{url_leukemia_statistic}.
Eine Leuk�mie macht sich in Proben des Knochenmarks dadurch bemerkbar, dass die Leukozytenzahl zugunsten anderer Zelltypen deutlich zunimmt, was Blutarmut und Probleme bei der Gerinnung zur Folge haben kann. F�r den Nachweis muss daher eine repr�sentative Anzahl an Zellen einer Klasse zugeordnet und gez�hlt werden. Dieser Vorgang ist zeitaufwendig und nicht standardisiert, weshalb es zu Abweichungen bei der Analyse durch verschiedene H�matologen kommen kann. Bei der wiederholten Auswertung durch den gleichen Arzt treten ebenfalls Diskrepanzen auf.
Das Ziel eines Projekts am Fraunhofer-Institut f�r Integrierte Schaltungen IIS ist es deshalb, den Prozess der Differentialz�hlung zu automatisieren, wodurch Ressourcen gespart werden k�nnen und eine bessere Vergleichbarkeit erzielt wird. Hierzu wurden Algorithmen entwickelt, um Zellen in gef�rbten Ausstrichen zu detektieren, zu segmentieren und zu klassifizieren. Diese Aufgabe wird dadurch erschwert, dass die Pr�parierung zwar Standards unterliegt, sich das Erscheinungsbild aber abh�ngig von Probendicke, Farbmenge, verwendetem Mikroskop usw. unterscheidet.
 
\section[Aufgabenstellung]{Aufgabenstellung}\label{sec:aufgabenstellung}
Das Ziel der vorliegenden Arbeit ist es, verschiedene Ans�tze zur Farbnormalisierung zu entwickeln, zu implementieren und miteinander zu vergleichen. Zun�chst soll hierf�r eine Literaturrecherche �ber g�ngige Verfahren sowie deren Eignung f�r das vorliegende Problem durchgef�hrt werden. Bestehende Verfahren sollen umgesetzt und gegebenenfalls angepasst und erweitert werden. Der Fokus der Evaluierung liegt auf dem Einfluss auf die Qualit�t der vorhandenen Segmentierungs- und Klassifikationsalgorithmen f�r Knochenmarkzellen. 

\chapter[Allgemeine Grundlagen]{Allgemeine Grundlagen}\label{sec:grundlagen}

\section[Mikroskopie]{Mikroskopie}\label{sec:mikroskopie}

\section[Knochenmarkdiagnostik]{}\label{sec:diagnostik}
\color{red}
Welche Informationen sind f�r den Arzt relevant. Wie w�rde ein nicht technisierter Ablauf aussehen
\color{black}
Die Gr�nde f�r eine zytologische Untersuchung des Knochenmarks sind vielf�ltig und reichen von au�erhalb der Norm liegenden Differentialblutausstrichen bis zum Verdacht auf Leuk�mie. Beim konventionellen Vorgehen wird zun�chst mit einer Punktionsnadel eine geringe Menge an Knochenmark entnommen. F�r die weitere Bearbeitung wird die Probe mit einer gerinnungshemmenden L�sung vermischt und auf einem Objekttr�ger ausgestrichen. Im Anschluss muss die Schicht eine Stunde trocknen und wird dann mit der F�rbung nach Pappenheim gef�rbt. Diese Aufgabe �bernimmt ein Automat, der daf�r ca. 45 Minuten ben�tigt. Einer der Farbstoffe ist hierbei Methylenblau, welcher gel�st positive Ladung tr�gt und somit negativ geladene Zellbestandteile f�rbt. Beim anderen Farbstoff handelt es sich um Eosin, das mit seiner negativen Ladung vor allem positive Proteinstrukturen der Zelle f�rbt. 

Die Analyse der Probe findet unter einem Lichtmikroskop statt. Bei zehnfacher Vergr��erung kann dabei zun�chst eine Aussage �ber Zelldichte und Fettgehalt getroffen werden. Teilweise gibt es auf dieser Stufe auch schon Informationen �ber qualitative Ver�nderungen, z.B. H�ufung bestimmter Zelltypen. Geeignete Bereiche des Ausstrich werden anschlie�end bei st�rkerer Vergr��erung (63x, seltener auch 100x) quantitativ ausgewertet. Daf�r werden die verschiedenen Zelltypen gez�hlt, wobei mindestens 200 Zellen ausgewertet werden sollten, um Aussagen treffen zu k�nnen. Diese manuelle Auswertung dauert in der Regel zwischen 5 und 15 Minuten, in Einzelf�llen sogar bis zu 30 Minuten. Das Ergebnis dieser Differentialz�hlung ist entscheidend f�r das weitere diagnostische und therapeutische Vorgehen. Hat ein Patient z.B. Leuk�mie, so wird dieses Verfahren eine deutlich erh�hte Leukozytenzahl aufdecken, welche die restliche Blutbildung verdr�ngt.

\section[Segmentierung]{Segmentierung}\label{sec:segmentierung}
\color{red}
Hier rein oder in Methoden? Hat ja nichts mit der Farbnormalisierung an sich zu tun
\color{black}

\section[Klassifizierung]{Klassifizierung}\label{sec:klassifizierung}
\color{red}
Hier rein oder Methoden? Hat ja nichts mit der Farbnormalisierung an sich zu tun?
\color{black}




\chapter[Stand der Technik]{Stand der Technik}\label{sec:stand_der_technik}


\chapter[Material]{Material}\label{sec:material}
\chapter[Methoden]{Methoden}\label{sec:methoden}

\section{Implementierung Khan und Macenko}\label{sec:implementierung_khan_macenko}

\section{Eigener Workflow}\label{sec:eigener_workflow}
\subsection{Bildclustering}\label{sec:clustering}

\subsection{Bspline}\label{sec:bspline}

\subsection{Optimierung der Stainvektoren}

\subsection{Auswahl Target}\label{sec:auswahl_target}

\subsection{Evaluierung}\label{sec:evaluierung}
\subsubsection{Segmentierung}\label{sec:eval_segmentierung}
\subsubsection{Klassifizierung}\label{sec:eval_klassifizierung}


\chapter[Ergebnisse]{Ergebnisse}\label{sec:ergebnis}
Im folgenden Kapitel sollen die Ergebnisse der Arbeit vorgestellt werden, wobei auf unterschiedliche Aspekte eingegangen wird. Neben einer quantitativen Analyse des Einflusses der verschiedenen Ans�tze auf Segmentierung und Klassifikation der Zellen, werden einige Resultate des Clustering gezeigt und verglichen. Zudem wird auf die Wirkung der Bestimmung der Stainvektoren durch die L�sung des Optimierungsproblems eingegangen. Zun�chst soll jedoch das verwendete Material genauer beleuchtet werden.

\section[Material]{Material}\label{sec:material}
Die f�r diese Arbeit verwendeten Knochenmarkpr�parate wurden nach Pappenheim gef�rbt und unter einem Hellfeldmikroskop untersucht. Die Bilder wurden dabei mit einer CCD-Kamera aufgenommen, die an das Mikroskop der Marke Zeiss, Modell Axio Imager Z2, angebracht wurde. Die Originalbilder haben dabei eine Gr��e von 2452x2056 Pixel, wobei die Pixelgr��e 3.45$\mu m$x 3.45$\mu m$ betr�gt. 
Als Eingang f�r die getesteten Normalisierungsalgorithmen die 400x400 Pixel gro�e Ausschnitte der Originale, welche auf dem ersten Teil des in Kapitel \ref{sec:seg_algo} beschriebenen Segmentierungsverfahren basieren. Jedes Bild hat genau eine Zielzelle, deren Zentrum optimalerweise mit dem Mittelpunkt des Bildes zusammenf�llt. Eine Handsegmentierung von ganzer Zelle und Zellkern steht zur Verf�gung und dient als Grundwahrheit f�r die Auswertung der Segmentierungsqualit�t. Zudem wurden alle Zellen durch einen Pathologen klassifiziert, so dass auch hierf�r eine Grundwahrheit vorhanden ist. F�r den Test der Segmentierung wurde ein Datensatz verwendet, welcher 1000 dieser 400x400 Teilbilder umfasst. Die durch die Normalisierung ver�nderten Bilder wurden im Anschluss als Trainingsdatensatz f�r die Klassifikation verwendet. Getestet wurde die Klassifikation anhand eines davon unabh�ngigen Sets mit 100 Bildern. 


\chapter[Diskussion]{Diskussion}\label{sec:diskussion}
Im Folgenden werden die Ergebnisse aus Kapitel \ref{sec:ergebnis} diskutiert und bewertet. Dabei wird zun�chst allgemein auf die Ans�tze zur Farbnormalisierung eingegangen, im Anschluss werden die beiden Bereiche Segmentierung und Klassifikation behandelt. 

\section{Allgemein}\label{sec:dis_allgemein}
Beim Vergleich des Farbeindrucks f�r die verschiedenen getesteten Methoden wird offensichtlich, dass Farbinformationen unterschiedlich stark verf�lscht werden. Dies soll in Abbildung \ref{fig:farbvergleich_norm} anhand von drei Beispielen gezeigt werden. Am deutlichsten ist die Ver�nderung bei Reinhard und bei der Anpassung von Mittelwert und Standardabweichung, welche sich nur im ge�nderten Farbraum unterscheiden. Als Zielbild diente dabei jeweils Referenzbild 1 (siehe Anhang \ref{app:reference}). Das Ergebnis ist dabei besonders stark durch den Anteil an gef�rbten Bereichen beeinflusst, was ein gro�er Nachteil der Verfahren ist. Weniger unnat�rlich wirken die Bilder nach Normalisierung mittels B-Spline und auf ein festgelegtes Pseudomaximum. Letztere Variante �hnelt im Hintergrund stark dem Original, da der dritte Kanal nicht angepasst wurde. Die gef�rbten Teile wirken insgesamt dunkler als beim Original, da das Maximum jeweils erh�ht wurde. Die mittels B-Splines angepassten Bilder normalisieren den Hintergrund sehr gut, so dass dieser sich in den verschiedenen Bildern stark �hnelt. Bei den gef�rbten Bereichen �ndert sich in Plasmabereichen teilweise das Verh�ltnis der beiden Farbstoffe. Zu sehen ist dieser Effekt beim Vergleich der Abbildungen \ref{fig:orig2} und \ref{fig:spline_giemsa2}. Bei der Zelle im Zentrum ist das Plasma nach der Anpassung leicht violett, w�hrend es zuvor bl�ulich erscheint. Die Referenz f�r beide F�lle ist Set 5 (siehe Anhang \ref{app:reference}), da hier die besten Ergebnisse f�r Segmentierung und Klassifikation erreicht werden. 

\newcommand{\mywidth}{0.13}
\begin{figure}[htb]
\center
\subfloat[Original 1]{
\includegraphics[width = \mywidth\textwidth]{pics/Anhang/B/02_or}}
\quad
\subfloat[B-Spline Giemsa 1]{
\includegraphics[width = \mywidth\textwidth]{pics/Anhang/B/02_gie}}
\quad
\subfloat[B-Spline Macenko 1]{
\includegraphics[width = \mywidth\textwidth]{pics/Anhang/B/02_mac}}
\quad
\subfloat[Festes Maximum 1]{
\includegraphics[width = \mywidth\textwidth]{pics/Anhang/B/02_fix}}
\quad
\subfloat[Mean/Std 1]{
\includegraphics[width = \mywidth\textwidth]{pics/Anhang/B/02_std}}
\quad
\subfloat[Reinhard 1]{
\includegraphics[width = \mywidth\textwidth]{pics/Anhang/B/02_rei}}
\quad
\subfloat[Original 2]{
\includegraphics[width = \mywidth\textwidth]{pics/Anhang/B/03_or}\label{fig:orig2}}
\quad
\subfloat[B-Spline Giemsa 2]{
\includegraphics[width = \mywidth\textwidth]{pics/Anhang/B/03_gie}\label{fig:spline_giemsa2}}
\quad
\subfloat[B-Spline Macenko 2]{
\includegraphics[width = \mywidth\textwidth]{pics/Anhang/B/03_mac}}
\quad
\subfloat[Festes Maximum 2]{
\includegraphics[width = \mywidth\textwidth]{pics/Anhang/B/03_fix}}
\quad
\subfloat[Mean/Std 2]{
\includegraphics[width = \mywidth\textwidth]{pics/Anhang/B/03_std}}
\quad
\subfloat[Reinhard 2]{
\includegraphics[width = \mywidth\textwidth]{pics/Anhang/B/03_rei}}
\quad
\subfloat[Original 3]{
\includegraphics[width = \mywidth\textwidth]{pics/Anhang/B/13_or}}
\quad
\subfloat[B-Spline Giemsa 3]{
\includegraphics[width = \mywidth\textwidth]{pics/Anhang/B/13_gie}}
\quad
\subfloat[B-Spline Macenko 3]{
\includegraphics[width = \mywidth\textwidth]{pics/Anhang/B/13_mac}}
\quad
\subfloat[Festes Maximum 3]{
\includegraphics[width = \mywidth\textwidth]{pics/Anhang/B/13_fix}}
\quad
\subfloat[Mean/Std 3]{
\includegraphics[width = \mywidth\textwidth]{pics/Anhang/B/13_std}}
\quad
\subfloat[Reinhard 3]{
\includegraphics[width = \mywidth\textwidth]{pics/Anhang/B/13_rei}}
\caption[Vergleich Ausgabebilder unterschiedlicher Methoden]{Die Abbildung zeigt drei Beispiele, wie sich die Normalisierung durch verschiedene Methoden auf Originalbilder auswirkt.\label{fig:farbvergleich_norm} }
\end{figure}

Der Grund f�r die gr��te Stabilit�t der Anpassung mittels B-Splines liegt in der Ber�cksichtigung der verschiedenartigen Bereiche im Bild. Prinzipiell w�re dies auch f�r die anderen Methoden m�glich, jedoch m�ssten hierf�r die �quivalenten Regionen separat aneinander angepasst werden. F�r das vorliegende Problem ist dies jedoch nicht m�glich, da die dadurch entstehenden Kanten, das Ergebnis des Segmentierungsalgorithmus vorweg nehmen w�rden. Bei der Anpassung durch einen B-Spline mit hohem Grad werden Kanten nicht unnat�rlich ver�ndert, da der B-Spline in hohem Ma�e stetig ist. Dadurch haben sich �hnelnde Pixel auch nach der Anpassung �hnliche Werte. Wird dagegen ein niedriger Grad gew�hlt, so haben eventuelle Ausrei�er ein starkes Gewicht. Im schlimmsten Fall kann dies dazu f�hren, dass der B-Spline nicht mehr monoton steigt. Ein Pixel der zuvor einen h�heren Wert hatte als ein anderer, k�nnte nach der Normalisierung eine niedrigere Intensit�t aufweisen. Da sich Eingangsbild und Referenz stark unterscheiden k�nnen, was zu Ausrei�ern bei den Kontrollpunkten f�hrt, muss ein hoher Grad gew�hlt werden um die Stabilit�t der Ergebnisse zu gew�hrleisten. 
 
\section{Segmentierung}\label{sec:dis_seg}
Bei Betrachtung der Ergebnisse der Segmentierung ist festzustellen, dass die Normalisierung f�r die meisten Zellen keinen Unterschied macht. Liegen beispielsweise mehrere Zellen dicht aufeinander, so dass keine deutliche Grenze mehr erkennbar ist, kann dieser Effekt durch die Anpassung nicht korrigiert werden. Sind dagegen alle Bereiche gut voneinander separierbar, so ist dies ebenfalls unabh�ngig von der Normalisierung und die automatische Segmentierung funktioniert f�r alle F�lle. Interessant sind die F�lle in der eine klare Verbesserung oder Verschlechterung vorliegt. Die Resultate legen nahe, dass eine positive Ver�nderung deutlich h�ufiger ist als eine negative. F�r die Anpassung mittels B-Splines und auch beim festen Maximum gilt, dass die Farben dazu tendieren dunkler zu werden. Zwar wird auch der Hintergrund etwas dunkler, jedoch wird eine leichte F�rbung eliminiert, welche sich nachteilig auf die Segmentierung auswirken kann. F�r eine gute Trennung zwischen Plasma und Kern ist es optimal, wenn beide Regionen m�glichst homogen sind und sich gleichzeitig voneinander abgrenzen lassen. Werden als Referenz Bilder gew�hlt, bei denen der gef�rbte Bereich diese Eigenschaften erf�llt, so k�nnen diese auch auf die anzupassenden Bilder �bertragen werden. Die Resultate in Kapitel \ref{sec:res_bspline_params} zeigen deutlich, dass nicht jedes Bild als Referenz geeignet ist, obwohl die automatische Segmentierung dort gut funktioniert. Es k�nnte z.B. sein, dass sich eine andere Zelle im Bild negativ auswirkt, oder st�rende Artefakte enthalten sind. In der vorliegenden Arbeit konnte eine Optimierung der Ergebnisse durch die Entfernung schwacher Einzelbilder aus dem Referenzset erzielt werden. Ein gewisse Vielfalt ist dennoch notwendig um die Eigenschaften m�glichst vieler Zellen abzubilden. Mit Set 6, welches nur drei Zellen enth�lt (siehe Anhang \ref{app:reference}), wurden die Ergebnisse wieder leicht schlechter. 
Wie zu erwarten war, erzielte auch die Skalierung auf ein festes Maximum bez�glich der Segmentierung gute Ergebnisse. Dies l�sst sich durch die Erh�hung des Kontrastes bei der Wahl eines verh�ltnism��ig hohen Maximums begr�nden. Eine Homogenisierung innerhalb Zellkern bzw. Plasma ist indes durch eine reine Skalierung nicht zu erreichen, was der Grund daf�r sein k�nnte, dass B-Splines f�r die Kernsegmentierung etwas besser funktionieren. 
Die Art einer Zelle ist von entscheidender Bedeutung f�r die automatische Segmentierung, sowie die Ver�nderung, welche mit einer m�glichen Anpassung einhergeht. Bei Normoblasten ist beispielsweise das Plasma relativ hell, was sie allgemein zu schwierigen F�llen f�r das Erkennen der ganzen Zelle macht. Nach der Farbnormalisierung wird dieses Ergebnis durchschnittlich sogar noch etwas schlechter, da im Clustering das Plasma h�ufig keiner Klasse zugeordnet werden kann. Dies hat zur Folge, dass sich stattdessen der Kontrast im Zellkern erh�ht. Daneben wird das Plasma noch heller und kann der Zelle nicht mehr zugeordnet werden. Durch den sehr dunklen Kern entspricht auch die Skalierung auf ein relativ hohes Pseudomaximum trotzdem einer Stauchung, wodurch das Plasma ebenfalls nochmals heller wird. Wird die ganze Zelle schon im Clustering-Schritt als gef�rbt erkannt, so besteht eine gute Chance, dass sich sowohl Zellkern als auch Plasma gut voneinander und vom Hintergrund absetzen. Gerade die Segmentierung des Zellkerns kann durch die angewandten Normalisierungsverfahren im Mittel verbessert werden. 

\section{Klassifikation}\label{sec:dis_klass}

F�r die Klassifikation wurde eine verh�ltnism��ig kleine, jedoch repr�sentative Datenbank von 1000 Trainings- und 100 Testbildern eingesetzt. Dies hat zur Folge, dass die Ergebnisse nicht mit jenen verglichen werden k�nnen die \citeeig{krappe2016automated} erzielt haben, da dort mit �ber 140000 Zellen gearbeitet wurde. Aufgrund der geringen Zahl an Bildern in dieser Arbeit k�nnen hier nur Trends abgeleitet werden, die noch best�tigt werden m�ssen. Eine valide Aussage ist jedoch, dass die Qualit�t der Klassifikation mit besserer Segmentierung steigt. Dieser Zusammenhang ist in Abbildung \ref{fig:seg_vs_class} dargestellt, in der die Segmentierung bez�glich Zellkern mit der Trefferquote und der Genauigkeit der Experimente in Verbindung gebracht ist. Eine verbesserte Trefferquote geht nicht zu Lasten der Genauigkeit, da beide Trendlinien ansteigen. Bei den beiden Ma�en f�r die Klassifikation sind  jeweils die Toleranzfelder ber�cksichtigt. Das Experiment mit der Handsegmentierung ist nicht mit einbezogen, da eine perfekte Segmentierung mit Wert 1.0 das Bild zu sehr beeinflussen und die restlichen Werte �berlagern w�rde. 

\begin{figure}
\center
\subfloat[Trefferquote]{
\includegraphics[width = 0.8\textwidth]{pics/Diskussion/seg_vs_class}}
\quad
\subfloat[Genauigkeit]{
\includegraphics[width = 0.8\textwidth]{pics/Diskussion/seg_vs_class_precision}}
\caption[Klassifikation in Abh�ngigkeit von der Segmentierung]{Abbildung \textbf{(a)} zeigt die Trefferwahrscheinlichkeit der Klassifikationsexperimente in Abh�ngigkeit von der Qualit�t der verwendeten Segmentierung. \textbf{(b)} �quivalent dazu die Genauigkeit in Abh�ngigkeit der Segmentierung. In blau sind dabei die Datenpunkte zu sehen, von denen jeder f�r ein Experiment steht. In schwarz ist jeweils die zugeh�rige lineare Trendlinie zu sehen.\label{fig:seg_vs_class}}
\end{figure}

Der Einfluss der Ver�nderung der Farbeeigenschaften der Zelle, kann nicht eindeutig bewertet werden. Die Experimente 1 und 10, in denen die gleiche automatische Segmentierung mit normalisierten bzw. originalen Bildern getestet wurde, lieferten vor allem unter Ber�cksichtigung der Toleranzfelder sehr �hnliche Resultate. Die Bef�rchtung, dass eine Anpassung aller verschiedenen Zellarten auf die gleichen Farbeigenschaften sich nachteilig auf die Klassifikationsf�higkeit auswirken k�nnte hat sich jedoch nicht best�tigt. Falls ein Trend in den vorliegenden Experimenten vorhanden ist, so ist er durch den Einfluss der Segmentierung �berlagert.  

\chapter[Zusammenfassung und Ausblick]{Zusammenfassung und Ausblick}\label{sec:ausblick}
W�hrend der Bearbeitung der vorliegenden Arbeit haben sich Erkenntnisse und Ideen ergeben, welche in der Zukunft eingearbeitet werden k�nnen. Diese sollen in den folgenden Abschnitten vorgestellt werden. Zun�chst wird die Arbeit jedoch kurz zusammengefasst.

\section{Zusammenfassung}
In der vorliegenden Arbeit wurden verschiedene Methoden zur Farbnormalisierung getestet und in Bezug auf die Auswirkung auf Segmentierung und Klassifikation bewertet. Das Standardverfahren nach Reinhard f�hrt die Anpassung einzeln in den Kan�len des sogenannten $l\alpha\beta$-Raum durch, in welchem die Kan�le m�glichst stark korrelieren. Dem gegen�bergestellt wurden Verfahren, welche auf Color Deconvolution basieren. Diese Farbraumtransformation hat f�r die vorliegende Anwendung den Vorteil, dass sie die Informationen �ber verwendete Farbstoffe aufgreift. Die Anpassung der so getrennten Kan�le erfolgte mittels B-Splines, einer Skalierung der Histogramme und der �bertragung von Mittelwert und Standardabweichung. Grundlage f�r die Color Deconvolution ist die Definition der Stainvektoren. Hierf�r wurden eine von \citeeig{macenko2009method} eingef�hrte automatische Methode sowie eigene Ideen, die auf der Definition eines Optimierungsproblems basieren, umgesetzt. Automatische Methoden sind fehleranf�llig solange sie nur auf kleinen Bildausschnitten arbeiten, weshalb die manuelle Bestimmung am sichersten ist. Bei der Segmentierung zeigten sich leichte Verbesserungen f�r die Betrachtung der ganzen Zelle, deutlicher war diese f�r die Trennung zwischen Kern und Plasma. F�r die meisten Zellen blieben die Segmentierungsqualit�t in einem �hnlichen Bereich, Ausrei�er waren jedoch h�ufiger verbessert als verschlechtert. Die Resultate der Klassifikation legen nahe, dass vor allem die genauere Segmentierung einen positiven Einfluss hat und die Resultate dadurch verbessert werden k�nnen. Inwieweit die Klassifikation durch die Ver�nderung der Farbeigenschaften beeinflusst wird, konnte aufgrund einer relativ kleinen Trainings- und Testdatenbank nicht abschlie�end gekl�rt werden. 

\section{Verwendung gr��erer Bildausschnitte}
Die quantitative Auswertung der Farbnormalisierung erfolgte in dieser Arbeit auf relativ kleinen Ausschnitten von 400$\times$400 Pixeln. Die Ursache lag dabei in einer fehlenden Grundwahrheit f�r die vorhandenen Ausschnitte mit einer Gr��e von 2452$\times$2056 Pixeln. Wie schon im vorhergehenden Kapitel \ref{sec:dis_klass} angemerkt, ist es problematisch die Farbeigenschaften von verschiedenartigen Zellen aufeinander abzubilden. Dieser Effekt kann mit der Verwendung gr��erer Bilder zumindest reduziert werden, da sowohl f�r Eingangsbild- als auch Referenz eine gr��ere Zahl Zellen vorhanden ist. Die Daten basieren auf dem ersten Teil des von \citeeig{krappe2014lokalisierung} verwendeten Segmentierungsverfahren, in welchem die Zellen lokalisiert werden \cite{krappe2014lokalisierung}. Aus diesem Grund wurde in der vorliegenden Arbeit die Ver�nderung f�r den zweiten Teil analysiert. Die Ergebnisse dieser Arbeit legen nahe, dass ein positiver Einfluss von Farbnormalisierungsverfahren auch f�r den ersten Schritt m�glich ist. Dies muss in der Zukunft jedoch noch nachgewiesen werden. In Anhang \ref{app:big} sind Beispiele f�r die Anpassung mittels B-Splines auf gro�en Daten gezeigt.

\section{Segmentierungsalgorithmus}
Eine Color Deconvolution mit geeigneten Stainvektoren k�nnte dazu eingesetzt werden die automatische Segmentierung zu verbessern. Die durch das Verfahren von Macenko gefundenen Stains f�hren bei bestimmten Zellarten zu einer Aufteilung von Plasma und Kern in unterschiedliche Kan�le. In einem Grauwertbild, welches auf dem Verh�ltnis der Intensit�ten dieser Kan�le basiert, sind die beiden Bereiche gut identifizierbar. Auch der Gradient innerhalb der Kan�le muss jedoch beachtet werden, da es Zellarten gibt, bei denen Zellkern und -plasma die gleiche Farbe haben, jedoch mit unterschiedlichen Intensit�ten. Bei diesen Zellen bleibt das Verh�ltnis der Kan�le gleich. Die geringe Korrelation zwischen den einzelnen Kan�len nach dieser Farbraumtransformation, welche in Abschnitt \ref{sec:correlation} nachgewiesen wurde, hat in jedem Fall das Potential, die Trennung zwischen Kern und Plasma zu verbessern. 

\section{Klassifikation}
Anhand der Klassifikationsresultate dieser Arbeit k�nnen noch keine validen Aussagen dar�ber gemacht werden, inwieweit die Farbver�nderungen der einzelnen Zellen das Ergebnis beeinflussen (siehe Abschnitt \ref{sec:dis_klass}). Aus diesem Grund ist ein Test mit vergr��erter Datenbank unerl�sslich. Sollte sich eine Verbesserung der Klassifikationsergebnisse nur aufgrund der genaueren Segmentierung ergeben, die Farb�nderungen jedoch einen negativen Einfluss auf die Klassifikation haben, kann die neue Segmentierung mit den Originalbildern verbunden werden. Sowohl angepasste als auch unver�nderte Farbbilder sind in der Prozesskette verf�gbar, so dass zur Optimierung ein Austausch der Informationen m�glichst ist. 
%\include{chapters/Zusammenfasung}


% ---------- Anhang ----------
% Die Datei Appendix.tex wird automatisch erzeugt!

%\begin{appendix}
%\input{appendices/Anhang}
%\end{appendix}

% ----------------------------

% Erzeugt das Literaturverzeichnis
%\bibliographystyle{plain}
%\bibliography{Literatur}
\newpage
\phantomsection \label{bibliography}
\addcontentsline{toc}{chapter}{Literaturverzeichnis} 
\bibliographystyle{plainnat} %numbers, order: alphabetical
%\bibliographystyle{natdin} %numbers, order: alphabetical
\bibliography{bib/quellen}

% Erzeugt eine Seite mit der Erklaerung
\declaration{nutzung}

\end{document}
