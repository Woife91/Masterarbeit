\chapter[Methoden]{Methoden}\label{sec:methoden}

\section{Implementierung Khan und Macenko}\label{sec:implementierung_khan_macenko}

\section{Eigener Workflow}\label{sec:eigener_workflow}

Im den folgenden Abschnitten soll aufgezeigt werden, inwiefern das Vorgehen von \citeauthor{khan2014nonlinear} und \citeauthor{macenko2009method} abge�ndert wurde um die aufgetretenen Probleme zu umgehen.

\subsection[Eigene Color Deconvolution]{Eigene Color Deconvolution}\label{sec:own_cd}
In n�chster N�herung wurden Stainvektoren f�r die F�rbung nach Giemsa (Methylenblau und Eosin) verwendet, welche von \citeeig{landini2016} nach dem in Abschnitt \ref{sec:stains_f�rbung} vorgestellten experimentellen Verfahren ermittelt wurden. F�r den fehlenden dritten Vektor, wird u.a. in der Arbeit von Khan angemerkt, dass dieser optimalerweise orthogonal zu den anderen beiden sein soll. Au�er im Fall von Einheitsvektoren, w�rde dieser jedoch negative Komponenten enthalten, was im Modell die Folge h�tte, dass die Absorption teilweise negativ w�re. Als L�sung wurden zwei verschiedene Ans�tze getestet. Die ersten beiden Zeilen der Stainmatrix sind bekannt und \citeauthor{landini2016} berechnet die dritte Zeile der Stainmatrix, in dem er zun�chst die Spalten normalisiert, f�r den Fall dass die ersten beiden Komponenten nicht schon �ber $1.0$ liegen. Tritt dies doch ein, so wird das Element der dritten Zeile auf $0.0$ gesetzt. Im n�chsten Schritt wird die Zeile normalisiert und Elemente die gleich $0$ werden zur Fehlerreduktion auf einen Wert $\epsilon = 0.01$ gesetzt. Diese Methode wurde der eigenen Idee gegen�bergestellt, bei der zun�chst �ber das Kreuzprodukt ein echt orthogonaler Vektor bestimmt wird. Dabei werden beide  Szenarien abh�ngig vom Vorzeichen ber�cksichtigt. Negative Komponenten werden auf $0.0$ gesetzt. Derjenige Vektor der nach dieser Operation die gr��ere L�nge aufweist, wird gew�hlt und normalisiert und als dritter Stainvektor eingesetzt. Unabh�ngig von der Wahl des dritten Vektors ist die Approximation f�r die vorhandenen Daten legitim, da der dritte Kanal f�r beide F�lle in gef�rbten Bereichen keine Information aufweist.
Eine Idee, die Zerlegung bildspezifisch zu optimieren, ist die Definition einer Zielfunktion f�r ein Optimierungsproblem, wobei als Initialstelle die Giemsa-Farbstoffe dienen k�nnen. Die Funktion ist von vier Parametern abh�ngig, n�mlich den jeweils ersten beiden Elementen der Farbe repr�sentierenden Stainvektoren. Dies ist nur m�glich, falls nur zwei Grundfarbstoffe verwendet werden, da ansonsten der dritte Vektor nicht in Abh�ngigkeit der ersten beiden berechnet werden kann. Folgende Eigenschaften der Zerlegung wurden als m�gliche Komponenten der Zielfunktion identifiziert:
\begin{itemize}
\item{Entropie in Kanal 3: Der dritte Kanal soll vor allem den Hintergrund repr�sentieren, welcher keine relevanten Informationen beinhaltet. Die Entropie dient als Ma� f�r den Informationsgehalt einer Quelle}
\item{Energie Kanal 3: �hnlich wie im ersten Punkt geht es hierbei darum, der nicht farbrelevanten Komponente eine m�glichst untergeordnete Rolle zuzuordnen}
\item{Abstand zur Hauptebene: Angelehnt an Macenko, wurde ein Ma� eingef�hrt, dass den Abstand f�r gef�rbte Pixel zur Hauptebene, welche durch die ersten beiden Stainvektoren definiert wird verringert. Einfluss darauf, wie die beiden Vektoren innerhalb dieser Ebene liegen, hat dieses Ma� keinen.}

\end{itemize}
\subsection{Bildclustering}\label{sec:clustering}

\subsection{Bspline}\label{sec:bspline}

\subsection{Optimierung der Stainvektoren}

\subsection{Auswahl Target}\label{sec:auswahl_target}

\subsection{Evaluierung}\label{sec:evaluierung}
\subsubsection{Segmentierung}\label{sec:eval_segmentierung}
\subsubsection{Klassifizierung}\label{sec:eval_klassifizierung}

