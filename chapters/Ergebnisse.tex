\chapter[Ergebnisse]{Ergebnisse}\label{sec:ergebnis}
Im folgenden Kapitel sollen die Ergebnisse der Arbeit vorgestellt werden, wobei auf unterschiedliche Aspekte eingegangen wird. Neben einer quantitativen Analyse des Einflusses der verschiedenen Ans�tze auf Segmentierung und Klassifikation der Zellen, werden einige Resultate des Clustering gezeigt und verglichen. Zudem wird auf die Wirkung der Bestimmung der Stainvektoren durch die L�sung des Optimierungsproblems eingegangen. Zun�chst soll jedoch das verwendete Material genauer beleuchtet werden.

\section[Material]{Material}\label{sec:material}
Die f�r diese Arbeit verwendeten Knochenmarkpr�parate wurden nach Pappenheim gef�rbt und unter einem Hellfeldmikroskop untersucht. Die Bilder wurden dabei mit einer CCD-Kamera aufgenommen, die an das Mikroskop der Marke Zeiss, Modell Axio Imager Z2, angebracht wurde. Die Originalbilder haben dabei eine Gr��e von 2452x2056 Pixel, wobei die Pixelgr��e 3.45$\mu m$x 3.45$\mu m$ betr�gt. 
Als Eingang f�r die getesteten Normalisierungsalgorithmen die 400x400 Pixel gro�e Ausschnitte der Originale, welche auf dem ersten Teil des in Kapitel \ref{sec:seg_algo} beschriebenen Segmentierungsverfahren basieren. Jedes Bild hat genau eine Zielzelle, deren Zentrum optimalerweise mit dem Mittelpunkt des Bildes zusammenf�llt. Eine Handsegmentierung von ganzer Zelle und Zellkern steht zur Verf�gung und dient als Grundwahrheit f�r die Auswertung der Segmentierungsqualit�t. Zudem wurden alle Zellen durch einen Pathologen klassifiziert, so dass auch hierf�r eine Grundwahrheit vorhanden ist. F�r den Test der Segmentierung wurde ein Datensatz verwendet, welcher 1000 dieser 400x400 Teilbilder umfasst. Die durch die Normalisierung ver�nderten Bilder wurden im Anschluss als Trainingsdatensatz f�r die Klassifikation verwendet. Getestet wurde die Klassifikation anhand eines davon unabh�ngigen Sets mit 100 Bildern. 

