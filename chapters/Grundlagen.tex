\chapter[Allgemeine Grundlagen]{Allgemeine Grundlagen}\label{sec:grundlagen}

\section[Mikroskopie]{Mikroskopie}\label{sec:mikroskopie}

\section[Knochenmarkdiagnostik]{}\label{sec:diagnostik}
\color{red}
Welche Informationen sind f�r den Arzt relevant. Wie w�rde ein nicht technisierter Ablauf aussehen
\color{black}
Die Gr�nde f�r eine zytologische Untersuchung des Knochenmarks sind vielf�ltig und reichen von au�erhalb der Norm liegenden Differentialblutausstrichen bis zum Verdacht auf Leuk�mie. Beim konventionellen Vorgehen wird zun�chst mit einer Punktionsnadel eine geringe Menge an Knochenmark entnommen. F�r die weitere Bearbeitung wird die Probe mit einer gerinnungshemmenden L�sung vermischt und auf einem Objekttr�ger ausgestrichen. Im Anschluss muss die Schicht eine Stunde trocknen und wird dann mit der F�rbung nach Pappenheim gef�rbt. Diese Aufgabe �bernimmt ein Automat, der daf�r ca. 45 Minuten ben�tigt. Einer der Farbstoffe ist hierbei Methylenblau, welcher gel�st positive Ladung tr�gt und somit negativ geladene Zellbestandteile f�rbt. Beim anderen Farbstoff handelt es sich um Eosin, das mit seiner negativen Ladung vor allem positive Proteinstrukturen der Zelle f�rbt. 

Die Analyse der Probe findet unter einem Lichtmikroskop statt. Bei zehnfacher Vergr��erung kann dabei zun�chst eine Aussage �ber Zelldichte und Fettgehalt getroffen werden. Teilweise gibt es auf dieser Stufe auch schon Informationen �ber qualitative Ver�nderungen, z.B. H�ufung bestimmter Zelltypen. Geeignete Bereiche des Ausstrich werden anschlie�end bei st�rkerer Vergr��erung (63x, seltener auch 100x) quantitativ ausgewertet. Daf�r werden die verschiedenen Zelltypen gez�hlt, wobei mindestens 200 Zellen ausgewertet werden sollten, um Aussagen treffen zu k�nnen. Diese manuelle Auswertung dauert in der Regel zwischen 5 und 15 Minuten, in Einzelf�llen sogar bis zu 30 Minuten. Das Ergebnis dieser Differentialz�hlung ist entscheidend f�r das weitere diagnostische und therapeutische Vorgehen. Hat ein Patient z.B. Leuk�mie, so wird dieses Verfahren eine deutlich erh�hte Leukozytenzahl aufdecken, welche die restliche Blutbildung verdr�ngt.

\section[Segmentierung]{Segmentierung}\label{sec:segmentierung}
\color{red}
Hier rein oder in Methoden? Hat ja nichts mit der Farbnormalisierung an sich zu tun
\color{black}

\section[Klassifizierung]{Klassifizierung}\label{sec:klassifizierung}
\color{red}
Hier rein oder Methoden? Hat ja nichts mit der Farbnormalisierung an sich zu tun?
\color{black}



