\chapter[Stand der Technik]{Stand der Technik}\label{sec:stand_der_technik}
Das grunds�tzliche Ziel beim Einsatz von Algorithmen zur Farbnormalisierung ist, dass Objekte gleicher Farbe auch �hnliche Pixelwerte haben. Dies sollte unabh�ngig von der Beleuchtung im Bild sein. Beim Anwendungsfeld der Mikroskopie kommen als Faktoren zus�tzlich noch Probendicke und Farbstoffmenge hinzu. Im folgenden Kapitel soll ein �berblick �ber bestehende Verfahren gegeben werden, wobei der Schwerpunkt, die Normalisierung auf mikroskopischen Bildern, in Abschnitt \ref{sec:mikroskopie_normalisierung} behandelt wird. Zun�chst soll jedoch auf Methoden eingegangen werden, welche zwar oft eingesetzt werden, im vorliegenden Fall aufgrund der speziellen Bildakquisition jedoch ungeeignet sind. 
\section[Allgemeine Farbnormalisierung]{Allgemeine Farbnormalisierung}\label{sec:allgemeine_normalisierung}
\color{red}
Gray World Assumption, Farbkonstanz

Referenz auf MA Bindl?
\color{black}
Das menschliche Gehirn ist innerhalb gewisser Grenzen in der Lage, die Farbe eines Objektes unabh�ngig von dessen Beleuchtung zu bestimmen. Diese F�higkeit ist die Grundlage f�r eine Gruppe von Ans�tzen, die als Farbkonstanzalgorithmen bezeichnet werden.

Das menschliche Auge ist das Vorbild f�r den sogenannten LMS-Farbraum. Die drei Kan�le entsprechen dabei den unterschiedlichen Zapfen, die auf lang-, mittel- und kurzwelliges Licht reagieren. Nach den Erkenntnissen von Johannes von Kries werden die unterschiedlichen Signale unabh�ngig voneinander vom Gehirn angepasst. Dieser Zusammenhang wird in der folgenden Formel \ref{equ:kries} dargestellt.
\begin{equation}\label{equ:kries}
L_{a} = k_{L}L//
M_{a} = k_{M}M//
S_{a} = k_{S}S
\end{equation}
$L_{a}$, $M_{a}$ und $S_{a}$ sind hierbei die angepassten Signale. $k_{L}$, $K_{M}$ und $k_{S}$ sind hingegen die unabh�ngigen Skalierungsfaktoren.
F�r die Bestimmung dieser Faktoren gibt es unterschiedliche M�glichkeiten. Zum einen k�nnen sie die Inversen des angenommenen Wei�punktes der Szene sein, zum anderen als Inverse des jeweils maximalen Werts des Kanals.

Farbkonstanzalgorithmen


\section[Farbnormalisierung in der Mikroskopie]{Farbnormalisierung in der Mikroskopie}\label{sec:mikroskopie_normalisierung}

\subsection{Farbranking}\label{sec:farbranking}

\subsection[Farbtransfer nach Reinhard]{Farbtransfer nach Reinhard}\label{sec:reinhard}
\color{red}
Reinhard allgemein + Anpassung an Mikroskopie(Mikroskopische Trainingsbilder)
\color{black}

\subsection{Color Deconvolution}\label{sec:color_deconvolution}

\subsubsection{Stainvektoren f�r eine F�rbung}\label{sec:stains_f�rbung}
\subsubsection{Stainvektoren f�r ein Bild}\label{sec:stains_bild}
\paragraph{Macenko}\label{sec:macenko}
\paragraph{Khan}\label{sec:khan}




