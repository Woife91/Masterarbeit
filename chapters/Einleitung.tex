\chapter[Einleitung]{Einleitung}\label{sec:einleitung}

\section[Motivation]{Motivation}\label{sec:motivation}

Analysen von Knochenmark werden im klinischen Alltag in verschiedenen F�llen angefordert, z.B. dem Verdacht auf Leuk�mie, da es sich hierbei um eine Erkrankung des Knochenmarks handelt. Die Krankheit betrifft alle Altersgruppen, wobei zwischen den folgenden h�ufigsten Formen unterschieden wird, je nach Verlauf und urs�chlichem Zelltyp:
\begin{itemize}
\item{Akute myeloische Leuk�mie(AML)}
\item{Chronische myeloische Leuk�mie(CML)}
\item{Akute lymphatische Leuk�mie(ALL)}
\item{Chronische lymphatische Leuk�mie(CLL)}
\end{itemize}
Im Kindesalter tritt fast immer ein akuter Verlauf auf, wobei die ALL am h�ufigsten vorkommt. Unabh�ngig vom Alter bildet aber CLL mit �ber einem Drittel die gr��te Gruppe. Im Jahr 2012 lag die Zahl der Neuerkrankungen mit Leuk�mie bei ca. 12640, 5\% davon unter 15 Jahren. Eine �bersicht �ber das Alter der neuerkrankten Patienten zeigt Abbildung \ref{fig:leukemia_age}. Aus dieser Statistik geht ebenfalls hervor, dass M�nner h�ufiger betroffen sind als Frauen, vor allem im Erwachsenenalter.
\begin{figure}
\center
\includegraphics[width = 0.9\textwidth]{pics/Einleitung/leukaemie_statistic_age}
\caption[Neuerkrankungen Leuk�mie nach Alter und Geschlecht]{Die Graphik zeigt die Statistik der Neuerkrankungen mit Leuk�mie in Deutschland f�r die Jahre 2011 und 2012 nach Alter(horizontale Achse) und Geschlecht (Balkenfarbe). Die Zahlen sind dabei relativ je 100000 normiert.\cite{url_leukemia_statistic}\label{fig:leukemia_age}}
\end{figure}
Bei Kindern ist die Prognose, die mit einer 5-Jahres �berlebenschance von bis zu 90\% korreliert, deutlich g�nstiger als bei Erwachsenen, bei denen die Chancen nur mit 35-50\% angegeben werden. Eine dauerhafte Heilung ist m�glich, es gibt sie aber nur selten z.B. durch Stammzellentransplantation. �ber die Ursachen von Leuk�mie ist indes wenig bekannt. Als Risikofaktoren gelten ionisierende Strahlung, z.B. bei Strahlen- oder Chemotherapie, und der Umgang mit Chemikalien wie Benzol. Diese Faktoren treffen aber nur auf eine kleine Gruppe der Erkrankten zu, weshalb weitere Faktoren diskutiert werden, ohne dass diese bis jetzt nachgewiesen werden konnten. Hierzu geh�ren Ern�hrungsgewohnheiten und Lebensstil, vor allem beim chronischen Verlauf, ein Einfluss von Viren und ein ungen�gendes Training des Immunsystem bei Heranwachsenden \cite{url_leukemia_statistic}.
Eine Leuk�mie macht sich in Proben des Knochenmarks dadurch bemerkbar, dass die Leukozytenzahl zugunsten anderer Zelltypen deutlich zunimmt, was Blutarmut und Probleme bei der Gerinnung zur Folge haben kann. F�r den Nachweis muss daher eine repr�sentative Anzahl an Zellen einer Klasse zugeordnet und gez�hlt werden. Dieser Vorgang ist zeitaufwendig und nicht standardisiert, weshalb es Abweichungen bei der Analyse durch verschiedene Pathologen gibt. Bei der wiederholten Auswertung durch den gleichen Arzt kommt es auch hier zu Diskrepanzen.
Das Ziel eines Projekts am Fraunhofer Institut f�r integrierte Schaltungen ist es deshalb, den Prozess der Differentialz�hlung zu automatisieren, wodurch Ressourcen gespart werden k�nnen und eine bessere Vergleichbarkeit realisiert wird. Hierzu wurden Algorithmen entwickelt, um Zellen in gef�rbten Ausstrichen zu detektieren, zu segmentieren und zu klassifizieren. Diese Aufgabe wird dadurch erschwert, dass die Pr�parierung zwar Standards unterliegt, sich das Erscheinungsbild aber abh�ngig von Probendicke, Farbmenge, verwendetem Mikroskop usw. unterscheidet.
 
\section[Aufgabenstellung]{Aufgabenstellung}\label{sec:aufgabenstellung}
Das Ziel der vorliegenden Arbeit ist es, verschiedene Ans�tze zur Farbnormalisierung zu entwickeln, zu implementieren und miteinander zu vergleichen. Hierbei sollen der Einfluss auf die Qualit�t der vorhandenen Segmentierungs- und Klassifikationsalgorithmen f�r Knochenmarkzellen untersucht werden. Dies umfasst eine Literaturrecherche �ber g�ngige Verfahren zur Farbnormalisierung und deren Eignung f�r das vorliegende Problem. 
